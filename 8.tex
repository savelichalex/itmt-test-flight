\documentclass{article}
\usepackage[utf8]{inputenc}
\usepackage{amsmath}
\usepackage{multicol}

\begin{document}

If the sequence $\{a_{n}\}_{n=1}^{\infty}$ tends to limit $L$ as $n\rightarrow \infty$, then for any fixed number $M>0$, the sequence $\{Ma_{n}\}_{n=1}^{\infty}$ tends to the limit $ML$.
\begin{equation*}
(\forall \epsilon > 0)(\exists n \in \mathbb{N})(\forall m \geq n)[|a_{m}-L|<\epsilon] \tag*{(1)}
\end{equation*}
\begin{equation*}
(\forall \epsilon > 0)(\exists n \in \mathbb{N})(\forall m \geq n)[|Ma_{m}-ML|<\epsilon] \tag*{(2)}
\end{equation*}
Proof: By contradiction.
\newline
Suppose:
\begin{align*}
(\forall m \geq n)[|Ma_{m}-ML|&\geq \epsilon]\\
M|a_{m}-L|&\geq \epsilon \tag*{(By algebra)}
\end{align*}
By (1) we could find such $a_{m}-L$ that equal to $\frac{\epsilon}{2M}$.
\begin{align*}
    M*\frac{\epsilon}{2M}&\geq \epsilon\\
    \frac{\epsilon}{2}&\geq \epsilon
\end{align*}
But this is contradiction. Hence we could find n for (2).
\newline
QED.

\end{document}