\documentclass{article}
\usepackage[utf8]{inputenc}
\usepackage{amsmath}
\usepackage{multicol}

\begin{document}

Let $x$ be a prime number, then $(x-2,x,x+2)$ is a prime triple.
\newline
\newline
A property of twin primes says that all of them except $(3,5)$ are of the form $6n\pm1$, where $n \in \mathbb{Q}$.
\newline
\newline
That means that the least distance between two primes must be $6\pm1=4$. In a triple we have two twin primes $(x-2,x)$ and $(x,x+2)$. But the distance between them is less then 4. The only possible solution include exceptional pair $(3,5)$ and a pair $(6n-1,6n+1)$ with $n=1 \rightarrow (5,7)$. That is the prime triple $(3,5,7)$.
\newline
QED.

\end{document}